%!TEX TS-program = xelatex
\documentclass[]{friggeri-cv}
\usepackage{fontawesome}


\begin{document}
\header{Joshua}{Reed}{}

% In the aside, each new line forces a line break
\begin{aside}
  \section{Contact}
    \uhref{tel:19712755001}{\underline(971)275-5001}
    \uhref{mailto:joshuadreed@gmail.com}{joshuadreed@gmail.com}
 \section{Sites}
    \uhref{https://github.com/reedjosh}{\faGithub{} GitHub: reedjosh}
    \uhref{https://www.linkedin.com/in/joshuadreed}{\faLinkedin{} LinkedIn: joshuadreed}
    \uhref{https://stackoverflow.com/users/8960404/jayreed1}{\faStackOverflow{} stack{\bf{}overflow}: jayreed1}
  \skill{Technologies \\ \& Frameworks}
    \uhref{https://pandas.pydata.org/}{Pandas} \& \uhref{https://github.com/SciRuby/daru}{Daru}
    \uhref{https://tortoise-orm.readthedocs.io/en/latest/}{ORMs}
    \uhref{https://graphql.org/}{GraphQL} \& \uhref{https://graphene-python.org/}{Graphene}
    \uhref{https://swagger.io/}{Rest}
    \uhref{https://rubyonrails.org}{Rails} \& \uhref{https://fastapi.tiangolo.com/}{FastAPI}
  \skill{Languages}
    Python 
    SQL
    Ruby
    Shell
    noSQL
    \LaTeX{}
    C/C++
    Go
    JavaScript
    TCL
  \skill{Software Skills}
    Vim/VSCode
    Linux
    Version Control
    TDD \& Coverage
    Command Line Utilities
    Object Oriented Programming
\end{aside}



\section{Experience}
\small
    \job{Software Engineer --- Tapeout Technology Development}
         {Intel}
         {2017---Present}
         \begin{itemize}
           \item Helped build a \uhref{https://fastapi.tiangolo.com}{FastAPI} data API used to abstract old and new automation systems.
                 Integrated an async Oracle driver and PyODBC connector. Added an async ORM for which I built the Oracle backend (see project below). 
                 Implemented the GraphQL endpoint. Utilized the combination of GraphQL and the ORM to automate the optimization of SQL queries. 
                 Built several complicated queries for Rest endpoints. Participated in infrastructure and CI/CD decision making.
                 {\bf\footnotesize(Python, FastAPI, Pandas, SQL, GraphQL, Tortoise ORM, Swagger, UnixODBC, PostgreSQL)}
           \item Refactored a Rails \textasciitilde{}10k SLOC website to \textasciitilde{}7k SLOC primarily through 
                \uhref{https://en.wikipedia.org/wiki/Don%27t_repeat_yourself}{DRYing} up the code and utilizing DataFrames. Used above mentioned Data API to 
                integrate with replacement automation system.
                 The refactor also included a great deal more logic for new states introduced to the models,
                 bumped Rails 4\rightarrow{}6, moved from Sprokets to Webpack, included the addition of 
                 a local MongoDB startup and test script, and added 10s of regression tests where none existed before. {\bf\footnotesize(Ruby, Rails, Mongoid, Mongo, Shell, JavaScript, JQuery)}
           \item Supported and improved the use model of Electronics Design Automation tools. Included
                 manipulating UI code to add features and automation previously thought to
                 be only possible if added as a feature from the external vendor. {\bf\footnotesize(Python, SQL, Pandas, Mongo, TCL\&TK, CalibreMDPV)}
           \item Built numerous reports and command line and GUI driven automation tools. Integrated diverse data sources and 
                 overcame several barriers to automation including connecting to Sharepoint via Kerberos auth. {\bf\footnotesize(Ruby, Python, SQL, noSQL, Kerberos, Shell)}
           \item Part of the transition team to update our software infrastructure across over 100k hosts. For this I rewrote a batch
                 processing Linux adapter, updated many of our aging repositories, and installed several updated libraries from source.
           \item Participated in on-call rotation for ops and compute triage.
        \end{itemize}

    \job{Internship Technical Marketing Engineer}
         {Mentor Graphics}
         {Summer \& Fall, 2016}
         \begin{itemize}
           \item Created a regression test generation program which converts arbitrary 
                 graphs in node neighbor format to usable input for Mentor's Calibre tools. {\bf\footnotesize(Python, Pandas, Shell, TCL)}
        \end{itemize}

    \job{Internship Design Engineer}
         {Garmin AT}
         {Spring \& Summer, 2015}
         \begin{itemize}
           \item Redesigned an aerial receiver's signal demodulation logic achieving a 
                 $60\%$ reduction in FPGA resource utilization. {\bf\footnotesize(VHDL)}
           \item Built tool that decodes USB voltage readings using a finite state machine algorithm. {\bf\footnotesize(Python)}
           \item Built tool that generates VHDL project compilation order from source files. {\bf\footnotesize(Python)}
         \end{itemize}

\section{Project}  
    \project{Tortois ORM --- Oracle Backend}
         \begin{itemize}
           \item \uhref{https://tortoise-orm.readthedocs.io/en/latest/}{Tortoise ORM} is an asynchronous Object Relationship Mapper.
           \item To design a GraphQL endpoint compatible with async FastAPI that would optimize queries automatically I needed an async ORM.
                 Unfortunately, I also needed both PostgreSQL and Oracle support. Since the Oracle backend for Tortoise was the only
                 missing link, I built it. \uhref{https://github.com/reedjosh/tortoise-orm}{My fork} has since been merged into develop.
         \end{itemize} 

\section{Education}
    \textbf{B.Sc., Electrical and Computer Engineering} \\
    \textit{{Oregon State University}\hfill 2017 }

\end{document}
