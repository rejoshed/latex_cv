%!TEX TS-program = xelatex
\documentclass[]{friggeri-cv}

\begin{document}
\header{Joshua}{Reed}
       {Aspiring Electrical and Computer Engineer}


% In the aside, each new line forces a line break
\begin{aside}
  \section{Contact}
    \href{tel:19712755001}{(971)275-5001}
    \href{mailto:reedjosh@oregonstate.edu}{reedjosh@\\oregonstate.edu}
  \section{Sites}
    \href{https://github.com/reedjosh}{GitHub: reedjosh}
    \href{https://www.linkedin.com/in/joshuadreed}{LinkedIn: joshuadreed}
  \section{Languages}
    Python, C, C++, VHDL, System Verilog, TCL, Bash, \LaTeX
  \section{Hardware Skills}
    FPGA and Microcontroller System Design,  Lab Tool Usage,  Basic Analog Signal and Power Supply Design, Schematic and PCB Design
  \section{Software Skills}
    Version Control, QA Concepts,
    Object Oriented Programming, Vim, Unix, Command Line Utilities
  \section{Others} Troubleshooting and Problem Solving, Communication, Teamwork
\end{aside}



\section{Education}

    \textbf{B.Sc., Electrical and Computer Engineering} \\
    Oregon State University \\
    Graduation: June, 2017 \\
    Current GPA: 3.47

\section{Experience}

    \job {Intern Technical Marketing Engineer}
         {Mentor Graphics}
         {June--November, 2016}
         \begin{itemize}
           \item Created a regression test generation program which converts arbitrary 
                 graphs in node neighbor format to usable input for Mentor's Calibre tools.
           \item Reproduced a customer bug using the test generation program I wrote
                 without the \\ customer's proprietary data.
           \item Used the test generation program for black box random and corner case testing.
        \end{itemize}

    \job {Digital Design Teaching Assistant}
         {Oregon State University}
         {Spring, 2016}
         \begin{itemize}
           \item Designed volt-meter final lab project framework. This was written in 
                 System Verilog targeting a Lattice FPGA and communicated with an 
                 external ADC via SPI.
           \item Designed and delivered lectures on topics such as Karnaugh Maps, Registers,
                 and System Verilog.
         \end{itemize}

    \job {Internship Design Engineer}
         {Garmin AT}
         {March--September, 2015}
         \begin{itemize}
           \item Redesigned an aerial reciever's signal demodulation logic acheiving a 
                 $60\%$ reduction of logic usage while implementing new VHDL 
                 standards to develop cleaner, more abstracted and extensible code.
           \item Wrote a script which troubleshoots and decodes USB communications given 
                 voltage readings in CSV format.
           \item Created a script which generates a top down VHDL project compilation order 
                 given only the project source files.
           \item Built a circuit that multiplexes display signals and provides
                 a controlled current source for the devices backlight.
         \end{itemize}

    \job {Electrical Fundimentals Teaching Assistant}
         {Oregon State University}
         {Winter, 2015}
         \begin{itemize}
           \item Lectured for weekly recitations on topics such as nodal and mesh 
                 analysis and thevenin and norton equivalencies.
           \item Created and graded weekly quizes.
         \end{itemize}

    \job {Digital Design Teaching Assistant}
         {Oregon State University}
         {Fall, 2014}
         \begin{itemize}
           \item Guided lab sessions twice per week focusing on topics such as logic
                 gates, Verilog, and block diagrams.
           \item Graded all student lab projects for the term.
         \end{itemize}



\section{Projects}  

    \project {Senior Design High Field Pulse Magnet}
         \begin{itemize}
           \item Worked in a three person team to build a high field pulse magnet.
           \item Pulse magnet successfuly generated a field in excess of 20 MW and 
                 crushed quarters to the size of a dime using this massive magnetic field.
           \item Responsible for oversized voltage display, PCB, and safety power indicator.
         \end{itemize} 
    

    \project {VLSI System Design Course Projects}
         \begin{itemize}
           \item Worked with senior instructor Roger Traylor to integrate an Altera FPGA 
                 into OSU's VLSI System Design course.
           \item Created a PLL LED demo project, lectured on PLL implementation, of which 
                 the 
                 \href{https://drive.google.com/file/d/0B_A8ZVw8vDrfZHMtQ0pHdFRmeWM/view}{screencapture} 
                 is still linked from the 
                 \href{http://classes.engr.oregonstate.edu/eecs/spring2016/ece474/}{course website} 
                today.
           \item Built sine wave generator utilizing the FPGAs internal rom.
         \end{itemize}


\end{document}
